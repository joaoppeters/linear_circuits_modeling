\documentclass{standalone}

\usepackage{siunitx}
\usepackage{tikz}
\usepackage{circuitikz}


\begin{document}
	
	\begin{circuitikz}[american]
		
		% fonte de tensao
		\draw (-1., -1.65) to [vsourcesin, l=$V_s$] (-1., 2.5);
		
		% capacitor entre nó 1 e nó 2
		\draw (-1., 2.5) to [capacitor, *-*, l=$C_1$] (3., 2.5);
		
		% capacitor entre nó 3 e terra
		\draw (7., 2.5) to [capacitor, *-, l=$C_2$] (7., -1.65);
		
		% diodo entre nó 2 e terra
		\draw (3., -1.65) to [diode, l_=$D_1$] (3., 2.5);
		
		% diodo entre nó 2 e nó 3
		\draw (3., 2.5) to [diode, l=$D_2$] (7., 2.5);
		
		% ramo de terra
		\draw (-1., -1.65) to [short, -*] (3., -1.65) to [short] (7., -1.65);
		
		% elemento terra
		\draw (3., -1.65) -- (3., -1.75) node[/tikz/circuitikz/bipoles/length=0.75cm, ground]{};
		
		% nós do circuito
		\draw (2.8, -1.45) node{gr};
		\draw (-1., 2.75) node{1};
		\draw (3., 2.75) node{2};
		\draw (7., 2.75) node{3};
		
	\end{circuitikz}
	
\end{document}