\documentclass{standalone}

\usepackage{siunitx}
\usepackage{tikz}
\usepackage{circuitikz}


\begin{document}
	
	\begin{circuitikz}[american]
		
		% fonte de tensao
		\draw (-1., -1.65) to [vsourcesin, l=$V_s$] (-1., 2.5);  
		
		% capacitor entre nó 1 e nó 2
		\draw (-1., 2.5) to [short, *-] (0., 2.5);
		\draw (0., 3.25) to [short] (0., 1.75);
		\draw (0., 1.75) to [resistor, l=$R_{c_1}$] (2., 1.75);
		\draw (2., 3.25) to [isource, l_=$J_{c_1}$] (0., 3.25);
		\draw (2., 3.25) to [short] (2., 1.75);
		\draw (2., 2.5) to [short, -*] (3., 2.5);
		
		% capacitor entre nó 3 e terra
		\draw (7., 2.5) to [short, *-] (7., 1.5);
		\draw (6.25, 1.5) to [short] (7.75, 1.5);
		\draw (6.25, 1.5) to [resistor, l=$R_{c_2}$] (6.25, -0.5);
		\draw (7.75, -0.5) to [isource, l_=$J_{c_2}$] (7.75, 1.5);
		\draw (7.75, -0.5) to [short] (6.25, -0.5);
		\draw (7., -0.5) to [short] (7., -1.65);
		
		% capacitor entre nó 4 e 2
		\draw (3., 7.5) to [short, *-] (3., 6.5);
		\draw (2.25, 6.5) to [short] (3.75, 6.5);
		\draw (2.25, 6.5) to [resistor, l_=$R_{c_3}$] (2.25, 4.5);
		\draw (3.75, 4.5) to [isource, l=$J_{c_3}$] (3.75, 6.5);
		\draw (2.25, 4.5) to [short] (3.75, 4.5);
		\draw (3., 4.5) to [short] (3., 2.5);
		
		% diodo entre nó 2 e terra
		\draw (3., 2.5) to [short] (3., 1.5);
		\draw (2.25, 1.5) to [short] (3.75, 1.5);
		\draw (2.25, 1.5) to [resistor, l=$R_{d_1}$] (2.25, -0.5);
		\draw (3.75, -0.5) to [isource, l_=$J_{d_1}$] (3.75, 1.5);
		\draw (3.75, -0.5) to [short] (2.25, -0.5);
		\draw (3., -0.5) to [short] (3., -1.65);
		
		% diodo entre nó 2 e nó 3
		\draw (3., 2.5) to [short] (4., 2.5);
		\draw (4., 3.25) to [short] (4., 1.75);
		\draw (4., 1.75) to [resistor, l=$R_{d_2}$] (6., 1.75);
		\draw (4., 3.25) to [isource, l=$J_{d_2}$] (6., 3.25);
		\draw (6., 3.25) to [short] (6., 1.75);
		\draw (6., 2.5) to [short, -*] (7., 2.5);
		
		% diodo entre nó 3 e nó 4
		\draw (3., 7.5) to [short] (7., 7.5);
		\draw (7., 7.5) to [short] (7., 6.5);
		\draw (6.25, 6.5) to [short] (7.75, 6.5);
		\draw (6.25, 6.5) to [resistor, l_=$R_{d_3}$] (6.25, 4.5);
		\draw (7.75, 4.5) to [isource, l=$J_{d_3}$] (7.75, 6.5);
		\draw (6.25, 4.5) to [short] (7.75, 4.5);
		\draw (7., 4.5) to [short] (7., 2.5);
		
		% ramo de terra
		\draw (-1., -1.65) to [short, -*] (3., -1.65) to [short] (7., -1.65);
		
		% elemento terra
		\draw (3., -1.65) -- (3., -1.75) node[/tikz/circuitikz/bipoles/length=0.75cm, ground]{};
		
		% nós do circuito
		\draw (2.8, -1.45) node{gr};
		\draw (-1., 2.75) node{1};
		\draw (2.8, 2.75) node{2};
		\draw (7.2, 2.75) node{3};
		\draw (3., 7.75) node{4};
		
	\end{circuitikz}
	
\end{document}