\documentclass{standalone}

\usepackage{siunitx}
\usepackage{tikz}
\usepackage{circuitikz}


\begin{document}
	
	\begin{circuitikz}[american]
		
		% fonte de tensao
		\draw (-1., -1.65) to [vsourcesin, l=$V_s$] (-1., 2.5);  
		
		% capacitor entre nó 1 e nó 2
		\draw (-1., 2.5) to [short, *-] (0., 2.5);
		\draw (0., 3.25) to [short] (0., 1.75);
		\draw (0., 1.75) to [resistor, l=$R_{c_1}$] (2., 1.75);
		\draw (2., 3.25) to [isource, l_=$J_{c_1}$] (0., 3.25);
		\draw (2., 3.25) to [short] (2., 1.75);
		\draw (2., 2.5) to [short, -*] (3., 2.5);
		
		% diodo entre nó 2 e terra
		\draw (3., 2.5) to [short] (3., 1.5);
		\draw (2.25, 1.5) to [short] (3.75, 1.5);
		\draw (2.25, 1.5) to [resistor, l=$R_{d_1}$] (2.25, -0.5);
		\draw (3.75, -0.5) to [isource, l_=$J_{d_1}$] (3.75, 1.5);
		\draw (3.75, -0.5) to [short] (2.25, -0.5);
		\draw (3., -0.5) to [short] (3., -1.65);
		
		% ramo de terra
		\draw (-1., -1.65) to [short, -*] (1., -1.65) to [short] (3., -1.65);
		
		% elemento terra
		\draw (1., -1.65) -- (1., -1.75) node[/tikz/circuitikz/bipoles/length=0.75cm, ground]{};
		
		% nós do circuito
		\draw (1., -1.45) node{gr};
		\draw (-1., 2.75) node{1};
		\draw (3., 2.75) node{2};
		
	\end{circuitikz}
	
\end{document}