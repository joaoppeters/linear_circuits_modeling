\documentclass{standalone}

\usepackage{siunitx}
\usepackage{tikz}
\usepackage{circuitikz}


\begin{document}
	
	\begin{circuitikz}[american]
		
		% fonte de tensao
		\draw (-1., -1.65) to [vsourcesin, l=$V_s$] (-1., 2.5);
		
		% capacitor entre nó 1 e nó 2
		\draw (-1., 2.5) to [capacitor, *-*, l=$C_1$] (3., 2.5);
		
		% capacitor entre nó 3 e terra
		\draw (7., 2.5) to [capacitor, *-, l=$C_2$] (7., -1.65);
		
		% capacitor entre nó 4 e nó 2
		\draw (3., 6.5) to [capacitor, *-, l_=$C_3$] (3., 2.5);
		
		% capacitor entre nó 5 e nó 3
		\draw (7., 6.5) to [capacitor, *-, l=$C_4$] (7., 2.5);
		
		% capacitor entre nó 6 e 4
		\draw (3., 10.5) to [capacitor, *-, l_=$C_5$] (3., 6.5);
		
		% capacitor entre nó 7 e nó 5
		\draw (7., 10.5) to [capacitor, *-, l=$C_6$] (7., 6.5);
		
		% diodo entre nó 2 e terra
		\draw (3., -1.65) to [diode, color=green!70!black, l_=$\textcolor{green!70!black}{D_1}$] (3., 2.5);
		
		% diodo entre nó 2 e nó 3
		\draw (3., 2.5) to [diode, color=red, l_=$\textcolor{red}{D_2}$] (7., 2.5);
		
		% diodo entre nó 3 e nó 4
		\draw (7., 2.5) to [diode, color=green!70!black, l_=$\textcolor{green!70!black}{D_3}$] (3., 6.5);
		
		% diodo entre nó 4 e nó 5
		\draw (3., 6.5) to [diode, color=red, l_=$\textcolor{red}{D_4}$] (7., 6.5);
		
		% diodo entre nó 5 e nó 6
		\draw (7., 6.5) to [diode, color=green!70!black, l_=$\textcolor{green!70!black}{D_5}$] (3., 10.5);
		
		% diodo entre nó 6 e nó 7
		\draw (3., 10.5) to [diode, color=red, l_=$\textcolor{red}{D_6}$] (7., 10.5);
		
		% resistor entre nó 7 e terra
		\draw (7., 10.5) to [resistor, l=$R_L$] (10., 10.5);
		
		% ramo de terra
		\draw (-1., -1.65) to [short, -*] (3., -1.65) to [short] (7., -1.65);
		
		% elemento terra
		\draw (3., -1.65) -- (3., -1.75) node[/tikz/circuitikz/bipoles/length=0.75cm, ground]{};
		\draw (10., 10.5) -- (10.1, 10.5) node[/tikz/circuitikz/bipoles/length=0.75cm, rotate=90, ground]{};
		
		% nós do circuito
		\draw (2.8, -1.45) node{gr};
		\draw (-1., 2.75) node{1};
		\draw (2.8, 2.75) node{2};
		\draw (7.2, 2.75) node{3};
		\draw (2.8, 6.75) node{4};
		\draw (7.2, 6.75) node{5};
		\draw (3., 10.75) node{6};
		\draw (7., 10.75) node{7};
		
	\end{circuitikz}
	
\end{document}